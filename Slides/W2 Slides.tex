\documentclass[compress]{beamer}
%\documentclass[handout]{beamer}

\mode<presentation>
{
  \usetheme{CambridgeUS}      % or try Darmstadt, Madrid, Warsaw, ...
  \usecolortheme{default} % or try albatross, beaver, crane, ...
  \usefonttheme{default}  % or try serif, structurebold, ...
  \setbeamertemplate{navigation symbols}{}
  \mode<beamer>{\setbeamertemplate{blocks}[rounded][shadow=true]}
  \setbeamertemplate{caption}[numbered]
  \useoutertheme{infolines}
  \useoutertheme[subsection=false]{miniframes}
} 

\usepackage[english]{babel}
\usepackage[utf8x]{inputenc}
\usepackage{pifont}
\usepackage{amssymb}
\usepackage{xcolor}
\usepackage{tikz}
\newcommand{\xmark}{\ding{55}}%
\usepackage{eurosym}
\usepackage{graphicx}
% set colors
\definecolor{myNewColorA}{RGB}{0, 45,114}
\definecolor{myNewColorB}{RGB}{0, 45,114}
\definecolor{myNewColorC}{RGB}{0, 45,114} % {130,138,143}
\setbeamercolor*{palette primary}{bg=myNewColorC}
\setbeamercolor*{palette secondary}{bg=myNewColorB, fg = white}
\setbeamercolor*{palette tertiary}{bg=myNewColorA, fg = white}
\setbeamercolor*{titlelike}{fg=myNewColorA}
\setbeamercolor*{title}{bg=myNewColorA, fg = white}
\setbeamercolor*{item}{fg=myNewColorA}
\setbeamercolor*{caption name}{fg=myNewColorA}
\setbeamercolor{date in head/foot}{fg=white}
\setbeamercolor{page number in head/foot}{fg=white}


\titlegraphic{%
\vspace{0cm}
    \includegraphics[width=4cm]{logo-vertical2.png}
}

\usepackage{subcaption}
% \usepackage{mathrsfs}

\usepackage{dirtytalk}
\usepackage{tcolorbox}
\usepackage{multicol}
\usepackage{multirow}
\usepackage{caption}
\usepackage{threeparttable}
\usepackage{pdfpages}
\usepackage{longtable}
\usepackage{adjustbox}
\usepackage{colortbl}
\usepackage{tikz}
\def\checkmark{\tikz\fill[scale=0.4](0,.35) -- (.25,0) -- (1,.7) -- (.25,.15) -- cycle;} 
\usepackage{pdfpages}

\usepackage{accents}
\newcommand{\ubar}[1]{\underaccent{\bar}{#1}}
%\usepackage{enumitem}

%% References - BEGIN
\usepackage[backend=bibtex8,style=authoryear-icomp,doi=false,url=false,isbn=false,eprint=false]{biblatex}
\renewbibmacro{in:}{}		% gets rid of the 'In' in front of the journal name
%\bibliographystyle{natbib}        
% \bibliography{references.bib}

% diagram (tree)
\usepackage{tikz}
\tikzset{
  treenode/.style = {shape=rectangle, rounded corners,
                     draw, align=center,
                     top color=white,
                     bottom color=blue!20},
  root/.style     = {treenode, font=\Large,
                     bottom color=red!30},
  env/.style      = {treenode, font=\ttfamily\normalsize},
  dummy/.style    = {circle,draw}
}

\title{Elements of Microeconomics: \\
       Discussion Section 1} 
\author{Rudy Kretschmar}
\date{}


\begin{document}

\begin{frame}
  \titlepage
\end{frame}


\begin{frame}{Outline}

    Chapter 2: Thinking like an economist

    \begin{itemize}
        \item How are economists like scientists? 
        \item How are economists not like scientists?
    \end{itemize}
    
    \medskip 

    Economists work with models; sometimes complicated though ideally simple. We'll start by looking at two foundational models. 

    \medskip
    
    Chapter 3: Interdependence and Gains from Trade

    \begin{itemize}
        \item What is comparitive advantage? Absolute advaantage?
        \item How do these generate the benefits of specialization and trade?
    \end{itemize}
\end{frame}


\section{Chapter 2: Thinking like an economist}



\begin{frame}
    \frametitle{Circular flow}
    \centering
    \includegraphics[width = 0.5\textwidth,keepaspectratio]{circular_flow.png}
\end{frame}

\begin{frame}
    \frametitle{Production possibility frontier}
    \centering
    \includegraphics[width = 0.6\textwidth,keepaspectratio]{ppf_example1.png}
\end{frame}

\begin{frame}{PPF of a firm}
    Let's start an Italian restaurant that makes pizzas and sandwiches.
    \begin{enumerate}
        \item What will our production possibility frontier look like?
        \item Why will it take the shape that it has?
        \item How can we read the opportunity cost? Does it matter which part of the PPF we look at?
        \item Why might the shape change over time?
    \end{enumerate}
\end{frame}

\begin{frame}{CPF of an individual}
    Now suppose we are \textit{going} to the Italian restaurant with a group of friends, and we want to decide what to order. Pizzas are \$10, sandwiches are \$5, and we have \$100 to spend.
    \begin{enumerate}
        \item What will our consumption possibility frontier look like?
        \item What is the opportunity cost of a pizza? Does it matter where on the CPF we are?
        \item What will happen to the CPF if we have \$200 to spend?
        \item What will happen to the CPF if the price of sandwiches increases to \$10?
    \end{enumerate}
\end{frame}

\begin{frame}{Circular-flow diagram of the Italian restaurant economy}
    Suppose the entire economy consists of Italian restaurants: during half of the week we work in one, and during the other half we buy food from them.

    \medskip

    \begin{itemize}
        \item What does the circular-flow diagram look like?
        \item What is missing from our model?
    \end{itemize}

\end{frame}

\begin{frame}
    \frametitle{Economists, economics, and economic reality}    
    Economists are often asked to guide economic policy.
    \begin{itemize}
        \item What are positive and normative statements?
        \item Why might two economists make different suggestions?
        \item Why might politicians ignore economists' suggestions?
    \end{itemize}
\end{frame}


\section{Interdependence and the Gains from Trade}

\begin{frame}{Thinking at the margin}
Something we glossed over in last week's discussion:

\begin{block}{}
    \begin{center}
        \textit{Economists think at the margin.}
    \end{center}
\end{block}

    More importantly, we believe that \textbf{firms and individuals do the same.
}
    \medskip
    \medskip

    This means when evaluating a decision, we think about what a small change in behavior will do to an outcome.

\end{frame}

% How people make decisions
\begin{frame}{Absolute advantage}
\begin{block}{}
    \textit{Absolute advantage} describes the ability to produce more of a good given a fixed quantity of inputs.
\end{block}
    \medskip

Let's consider two restaurants: Franco's Trattoria and Grano Pasta. Both of them can produce two dishes: salads and pasta. Given 1000 minutes of labor time, they can produce the following amounts of each dish:

    \centering
    \begin{table}
    \begin{tabular}{|c|c|c|}
      \hline
      \textbf{Restaurant} & \textbf{Pasta} & \textbf{Salads} \\
      \hline
      \textbf{Grano} & 100 & 20 \\
      \hline
      \textbf{Franco's} & 200 & 100 \\
      \hline
    \end{tabular}
  \end{table}

    What is their (*cough, marginal) cost, in minutes, to produce pasta and salads?
\end{frame} 

\begin{frame}{Absolute advantage}
    Assume that there is a \textit{constant transferability} from one dish to the other:
    \begin{enumerate}
        \item Draw the production possibility frontiers for the two restaurants.
        \item Who has the absolute advantage in producing pasta?
        \item Who has the absolute advantage in producing salads?
    \end{enumerate}

\end{frame}

\begin{frame}{Comparative advantage}
   Before we discuss comparative advantage, let's think about the opportunity cost of each firm for each dish:
   \begin{enumerate}
    \item What are the slopes of the two PPFs?
    \item What is Grano's opportunity cost for producing pasta and salads?
    \item What is Franco's opportunity cost for producing pasta and salads?
   \end{enumerate}
   In other words: what is the \textit{trade-off} that each restaurant faces as they change their production from one dish to another?
\end{frame} 

\begin{frame}{Comparative advantage}
    The \textit{opportunity cost} of producing salads is the amount of pasta they could have produced with the same input. In our example, this is constant.

    \medskip
    
\begin{block}{}
    A restaurant has a \textit{comparative advantage} in producing pasta compared to their competitor if their opportunity cost is lower.
\end{block}

    \medskip

    \begin{enumerate}
        \item Can a firm have an absolute advantage in both goods?
        \item Can a firm have a comparative advantage in both goods?
        \item What is the relationship between the comparative advantage in good A and good B?
    \end{enumerate}
\end{frame} 

\begin{frame}{Comparative advantage}
\begin{itemize}
    \item   The comparative advantage in producing good A is the \textit{inverse} of the comparative advantage in producing good B.
    \item     If the comparative advantage in good A is high, the comparative advantage for good B must be low.

\end{itemize}

    \medskip

    Comparative advantage depends on the \textit{opportunity cost}: these concepts are linked.
\end{frame} 

\begin{frame}{Comparative advantage}
    Since most customers like to order a salad with their pasta, Franco's and Grano both want to offer both salads and pasta (not necessarily in equal quantities).

    \medskip

    If both spend half their resources on each dish, what is their output?

    \medskip

    Now suppose the two restaurants can trade with each other. What is one set of productions, and one possible trade, which would leave them both better off?

\end{frame} 

\begin{frame}{Comparative advantage}
    When they both split their 1000 minutes 50/50 between the two dishes, their output is:

    \begin{table}
    \begin{tabular}{|c|c|c|}
      \hline
      \textbf{Restaurant} & \textbf{Pasta} & \textbf{Salads} \\
      \hline
      \textbf{Grano} & 50 & 10 \\
      \hline
      \textbf{Franco's} & 100 & 50 \\
      \hline
      \textbf{Total output} & 150 & 60 \\
      \hline
    \end{tabular}
    \caption{50/50 split}
  \end{table}

  Now suppose the two restaurants can trade with each other. What is one set of productions, and one possible trade, which would leave them both better off?
\end{frame} 

\begin{frame}{Possible trade}
    There are many possible answers to this last question, but let's go back to our principle at the beginning of the discussion, and \textit{think at the margin}.
    \begin{itemize}
        \item Grano produces 1 fewer salads and 5 more steaks
        \item Franco's produces 2 fewer steaks, and 1 more salad
    \end{itemize}
    Then their production is:

    \begin{table}
    \begin{tabular}{|c|c|c|}
      \hline
      \textbf{Restaurant} & \textbf{Pasta} & \textbf{Salads} \\
      \hline
      \textbf{Grano} & 55 & 9 \\
      \hline
      \textbf{Franco's} & 98 & 51 \\
      \hline
      \textbf{Total output} & 153 & 60 \\
      \hline
    \end{tabular}
    \caption{Possible trade}
  \end{table}

    Total production has gone up!
\end{frame} 

\begin{frame}{Possible trade}
    Which trade would leave them both better off?

    \medskip

    Say Grano trades 3 pastas to Franco's in exchange for one salad:

    \begin{table}
    \begin{tabular}{|c|c|c|}
      \hline
      \textbf{Restaurant} & \textbf{Pasta} & \textbf{Salads} \\
      \hline
      \textbf{Grano} & 52 & 10 \\
      \hline
      \textbf{Franco's} & 101 & 50 \\
      \hline
    \end{tabular}
    \caption{Gains of trade}
  \end{table}

    They both have the same amount of salads as before, but more steaks! So we can say that they are each better off.

    \medskip
    
    Should they continue to specialize?

\end{frame} 

\begin{frame}{Price of trade}
    Here we just asserted a trade that would make both parties better off in terms of the amount of each dish. But how can we know both parties will agree to the trade?
    
    \medskip

    This is determined by the price of each good. In the example we gave, the ``price'' of one salad was 3 steaks.

    \begin{enumerate}
        \item What if the price of 1 salad was 3.5 pastas?
        \item What if the price of 1 salad was 1 pasta?
        \item What if the price of 1 salad was 6 pasta?
    \end{enumerate}
\end{frame}

\begin{frame}{Price of trade}
    The first example would still leave both parties better off, but the second two would not.

    \medskip

    We are not ready yet to discuss where prices come from, but we do have a general rule:

    \begin{center}
        \textit{For trade to make both parties better off, the price must lie between the two opportunity costs.}
    \end{center}

\end{frame}

\begin{frame}{Discussion questions}

\begin{enumerate}
    \item Should Kevin Durant wash his own car?
    \item Should the U.S. trade with other countries?
    \item Should a chef build his own house?
    \item Should I make my own clothes?
    \item Should you be teaching this class?
\end{enumerate}

    \medskip

    \begin{block}{The main takeaway from this chapter:}
        \textit{Due to comparative advantage, specialization and trade can leave everyone participating better off.}
    \end{block}

\end{frame} 


\end{document}
     


